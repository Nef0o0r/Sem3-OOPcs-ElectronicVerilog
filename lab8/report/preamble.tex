\documentclass[a4paper, 14pt, oneside, openany]{memoir}

%%% Поля, отступы и межстрочный интервал %%%

\linespread{1.5} % Междустрочный 1.5
\usepackage[left=20mm, right=15mm, top=20mm, bottom=20mm]{geometry} % Поля
\pagestyle{plain} % Убираем верхние колонтитулы
\parindent=1.25cm % Абзацный отступ 1.25 см
\usepackage{indentfirst} % Отступ для первого абзаца
\usepackage{fancyhdr} % Футеры и хедэры
\fancypagestyle{plain}{ % Переназначение стиля
\fancyhf{} % Очистка всех полей
\fancyfoot[C]{\fontsize{12pt}{12pt}\selectfont\thepage} % Номер страницы
\renewcommand{\headrulewidth}{0pt} % Высота хедера
\renewcommand{\footrulewidth}{0pt}} % Высота футера
\pagestyle{plain} % Применение стиля

%%% Язык и шрифт %%%
\usepackage[english, russian]{babel} % Русский язык
\usepackage{fontspec}
\setmainfont{Times New Roman} % Шрифт текста

%%% Заголовки и подзаголовки %%%

\usepackage{titlesec}
%\titleformat{\chapter}{\normalfont\bfseries\centering\MakeUppercase}{\thechapter}{8pt}{}
\titleformat{\chapter}{\normalfont\bfseries\centering\MakeUppercase}{}{8pt}{} % Без нумерации и модержания
\titleformat{\section}{\normalfont\bfseries}{\thesection}{8pt}{}
\titleformat{\subsection}{\normalfont\bfseries}{\thesubsection}{8pt}{}
\titleformat{\subsubsection}{\normalfont\bfseries}{\thesubsubsection}{8pt}{}
\titlespacing*{\chapter}{0pt}{-26pt}{0pt}
\titlespacing*{\section}{\parindent}{0pt}{0pt}
\titlespacing*{\subsection}{\parindent}{0pt}{0pt}
\titlespacing*{\subsubsection}{\parindent}{0pt}{0pt}
\setsecnumdepth{subsection} % Разделы считать до третьего уровня

%%% Оглавление %%%

%\renewcommand*{\chapternumberline}[1]{} % Чтобы номер обычной главы не печатался но считался
\renewcommand*{\chapterheadstart}{} % Нет отступа над заголовком
\renewcommand*{\printchaptertitle}{\normalfont\bfseries\centering\MakeUppercase} % Стиль заголовка
\addto\captionsrussian{\renewcommand\contentsname{Содержание\vspace{-20pt}}} % Меняем слово "Оглавление" на "Содержание"
% \setrmarg{2.55em plus1fil} % Запрещаем переносы слов в оглавлении
\setlength{\cftbeforechapterskip}{0pt} % Нет интервала между заголовками глав
\renewcommand*{\cftchapternumwidth}{1.5em} % Разделитель между номером главы и заголовком
\renewcommand*{\cftchapterfont}{\normalfont\MakeUppercase} % Шрифт глав
\renewcommand*{\cftchapterpagefont}{\normalfont} % Шрифт страниц
\renewcommand*{\cftchapterdotsep}{\cftdotsep} % Точки до номера страницы после названий глав
\renewcommand*{\cftdotsep}{1} % Задаем расстояние между точками
\renewcommand*{\cftchapterleader}{\cftdotfill{\cftchapterdotsep}} % Точки стандартной формы
\maxtocdepth{subsection} %Только разделы первыхтрех уровней

%%% Выравнивание и переносы %%%

% http://tex.stackexchange.com/questions/241343/what-is-the-meaning-of-fussy-sloppy-emergencystretch-tolerance-hbadness
% http://www.latex-community.org/forum/viewtopic.php?p=70342#p70342
\tolerance 1414
\hbadness 1414
\emergencystretch 1.5em
\hfuzz 0.3pt
\vfuzz \hfuzz
%\dbottom
% \sloppy % Избавляемся от переполнений
\clubpenalty=10000 % Запрещаем разрыв страницы после первой строки абзаца
\widowpenalty=10000 % Запрещаем разрыв страницы после последней строки абзаца
\brokenpenalty=4991 % Ограничение на разрыв страницы, если строка заканчивается переносом

%%% Буквы русского алфавита для перечислений %%%

\makeatletter
\def\russian@alph#1{\ifcase#1\or
	а\or б\or в\or г\or д\or е\or ж\or
	и\or к\or л\or м\or н\or
	п\or р\or с\or т\or у\or ф\or х\or
	ц\or ш\or щ\or э\or ю\or я\else\xpg@ill@value{#1}{russian@alph}\fi}
\makeatother

%%% Рисунки и таблицы %%%
\usepackage{cellspace} % Для работы титульника
\usepackage{float, listings, graphicx, caption, subcaption} % Работа с графикой и подписями
\setlength{\intextsep}{14pt} % Между текстом и рисунком
\graphicspath{{images/}} % Папка с рисунками
\captionsetup[figure]{font=small, width=\textwidth, name=Рисунок, justification=centering} % Подписи к рисункам
\captionsetup[subfigure]{font=small} % Индексы подрисунков
\captionsetup[table]{singlelinecheck=false, font=small, width=\textwidth, justification=justified} % Подписи к таблицам
\captiondelim{ --- } % Разделитель номера и подписи
\setkeys{Gin}{width=\textwidth} % По умолчанию размер под ширину текста
\renewcommand{\thesubfigure}{\asbuk{subfigure}} % Нумерация подрисунков
\usepackage[section]{placeins} % float не вылезают за границы секциии

%%% Ссылки и гиперссылки %%% 

\usepackage{hyperref}
\hypersetup{hidelinks, % Выключить цвета и рамки для ссылок
			linktoc=all} % В оглавлении ссылки всех уровней

%%% Списки %%%

\usepackage{enumitem}
\makeatletter
\AddEnumerateCounter{\asbuk}{\russian@alph} % Подключение Кириллицы
\makeatother

\newlist{particle}{enumerate}{2} % Пункты и подпункты как в ГОСТ
\setlist[particle]{align = left, nosep, leftmargin = 0pt, labelsep = 8pt, listparindent = \parindent, itemindent = \labelwidth + \labelsep + \parindent + \leftmargin}
\setlist[particle, 1]{label=\bfseries\thesection.\arabic*}
\setlist[particle, 2]{label*=\bfseries.\arabic*}

\setlist[enumerate]{label=\ \ \ \ \ \arabic*), align = left, nosep, leftmargin = 0pt, labelsep = 3.6pt, listparindent = \parindent, itemindent = \labelwidth + \labelsep + \parindent + \leftmargin - 17.9pt} % Нумерация цифрами

\newlist{letter}{enumerate}{1}
\setlist[letter]{label = \ \ \ \ \ \asbuk*), align=left, nosep, leftmargin=0pt, labelsep = 3.6pt, listparindent = \parindent, itemindent = \labelwidth + \labelsep + \parindent + \leftmargin - 17.9pt}  % Нумерация буквами

\setlist[itemize]{label = \ \ \ \ \ \normalfont{--}, align = left, nosep, leftmargin = 0pt, labelsep = 3.6pt, listparindent = \parindent, itemindent = \labelwidth + \labelsep + \parindent + \leftmargin - 17.9pt} % Перечисление

%%% Счетчики для нумерации объектов %%%

\counterwithout{figure}{chapter} % Сквозная нумерация рисунков
\counterwithout{equation}{chapter} % Сквозная нумерация формул
\counterwithout{table}{chapter} % Сквозная нумерация таблиц

%%% Библиография %%%

%\renewcommand\UrlFont{\rmfamily} % Шрифт для ссылок на сайты
\usepackage{csquotes} % Сложное цитирование
\usepackage[backend=biber, bibencoding=utf8, sorting=none, style=gost-numeric, language=auto, autolang=other, sortcites=true, movenames=false, maxnames=5, minnames=3, doi=false, isbn=false]{biblatex}[2016/09/17] % Оформление
\renewcommand*{\bibfont}{\normalfont\upshape\itshape\normalsize\rmfamily} % Шрифт
\DeclareDelimFormat{bibinitdelim}{} % Нет пробела между инициалами
\addbibresource{lib.bib} % Файл с библиографией
%%% Скрипт для русского языка
\DeclareSourcemap{\maps[datatype=bibtex]{\map{\step[fieldsource=title, match=\regexp{^\P{Cyrillic}*\p{Cyrillic}.*}, final] \step[fieldset=langid, fieldvalue={russian}]}\map{\step[fieldset=langid, fieldvalue={english}]}}}

%%% Доп функционал %%%

%\usepackage{amsmath} % Мат пакет

%%% Собственные функции %%%

\usepackage{xargs} % Много аргументов
\NewDocumentCommand\pic{O{ht}O{120mm}mm} % Вставка изображения
{
	\begin{figure}[#1]
		\centering
		\includegraphics[width=#2]{#3}
		\caption{#4}
		\label{pic:#3}
	\end{figure}
	\vspace{-15pt}
}
\newcommand{\picref}[1]  % Вставка ссылки на изображение
{
	(\hyperref[pic:#1]{рисунок \ref*{pic:#1}})
}

%%% Неформальный код %%%

\setmonofont{PT Mono}
% \usepackage{polyglossia}
% \setsansfont{Arial}
% \setmonofont{Courier New}
% \newfontfamily\cyrillicfont[Script=Cyrillic]{Times New Roman}
% \newfontfamily\cyrillicfontsf[Script=Cyrillic]{Arial}
% \newfontfamily\cyrillicfonttt[Script=Cyrillic]{Courier New}
% \usepackage{minted}
% \usemintedstyle[python]{pastie}
% \usemintedstyle[c]{pastie}
% \usemintedstyle[cpp]{pastie}
% \usemintedstyle[java]{pastie}
