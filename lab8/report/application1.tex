\chapter*{Приложение 1}

Файл CreateAccount.cs:
\begin{lstlisting}[style=CSharpStyle]
class CreateAccount
{
    // Test Harness
    static void Main() 
    {
		BankAccount acc1, acc2, acc3, acc4;

		acc1 = new BankAccount();
		acc2 = new BankAccount(AccountType.Deposit);
		acc3 = new BankAccount(100);
		acc4 = new BankAccount(AccountType.Deposit, 500);

		acc1.Deposit(100);
		acc1.Withdraw(50);
		acc2.Deposit(75);
		acc2.Withdraw(50);
        acc3.Withdraw(30);
		acc3.Deposit(40);
		acc4.Deposit(200);
		acc4.Withdraw(450);
		acc4.Deposit(25);

		Write(acc1);
		Write(acc2);
		Write(acc3);
		Write(acc4);
    }
      
    static void Write(BankAccount acc)
    {
        Console.WriteLine("Account number is {0}",
                acc.Number());
        Console.WriteLine("Account balance is {0}", 
                acc.Balance());
        Console.WriteLine("Account type is {0}",
                acc.Type());
		Console.WriteLine("Transactions:");
		foreach (BankTransaction tran in 
                acc.Transactions()) 
		{
			Console.WriteLine("Date/Time: {0}\tAmount: " +
                    "{1}", tran.When(), tran.Amount());
		}
		Console.WriteLine();
    }
}
\end{lstlisting}  

Файл BankAccount.cs:
\begin{lstlisting}[style=CSharpStyle]
using System.Collections;

class BankAccount 
{
	private long accNo;
    private decimal accBal;
    private AccountType accType;
	private Queue tranQueue = new Queue();
    
    private static long nextNumber = 123;

	// Constructors
    public BankAccount()
    {
        accNo = NextNumber();
        accType = AccountType.Checking;
	    accBal = 0;
    }

    public BankAccount(AccountType aType)
	{
		accNo = NextNumber();
		accType = aType;
		accBal = 0;
	}

	public BankAccount(decimal aBal)
	{
		accNo = NextNumber();
		accType = AccountType.Checking;
		accBal = aBal;
	}

	public BankAccount(AccountType aType, decimal aBal)
	{
		accNo = NextNumber();
		accType = aType;
		accBal = aBal;
	}

    public bool Withdraw(decimal amount)
    {
        bool sufficientFunds = accBal >= amount;
        if (sufficientFunds) {
            accBal -= amount;
			BankTransaction tran = new BankTransaction(
                    -amount);
			tranQueue.Enqueue(tran);
        }
        return sufficientFunds;
    }
    
    public decimal Deposit(decimal amount)
    {
        accBal += amount;
		BankTransaction tran = new BankTransaction(amount);
		tranQueue.Enqueue(tran);
        return accBal;
    }

	public Queue Transactions()
	{
		return tranQueue;
	}
    
    public long Number()
    {
        return accNo;
    }
    
    public decimal Balance()
    {
        return accBal;
    }
    
    public string Type()
    {
        return accType.ToString();
    }

    private static long NextNumber()
    {
        return nextNumber++;
    }
}
\end{lstlisting}  

Файл BankTransaction.cs:
\begin{lstlisting}[style=CSharpStyle]
/// <summary>
///   A BankTransaction is created every time a deposit or withdrawal occurs on a BankAccount
///   A BankTransaction records the amount of money involved, together with the current date and time.
/// </summary>
public class BankTransaction
{
    private readonly decimal amount;
    private readonly DateTime when;

    public BankTransaction(decimal tranAmount)
    {
       amount = tranAmount;
       when = DateTime.Now;
    }

    public decimal Amount()
    {
        return amount;
    }

    public DateTime When()
    {
        return when;
    }			 
}
\end{lstlisting}  

Файл AccountType.cs:
\begin{lstlisting}[style=CSharpStyle]
\begin{verbatim}
enum AccountType 
{ 
    Checking, 
    Deposit 
}
\end{lstlisting}  
\endinput
