\chapter*{Приложение 2}

Файл CreateAccount.cs:
\begin{lstlisting}[style=CSharpStyle]
class CreateAccount
{
    static void Main() 
    {
        BankAccount acc1;

        acc1 = new BankAccount();
        acc1.Deposit(200);
        acc1.Withdraw(100);
        Write(acc1);

        acc1.Dispose();
        GC.Collect();
        GC.WaitForPendingFinalizers();
    }
      
    static void Write(BankAccount acc)
    {
        Console.WriteLine("Account number is {0}",
                acc.Number());
        Console.WriteLine("Account balance is {0}", 
                acc.Balance());
        Console.WriteLine("Account type is {0}",
                acc.Type());
		Console.WriteLine("Transactions:");
		foreach (BankTransaction tran in 
                acc.Transactions()) 
		{
			Console.WriteLine("Date/Time: {0}\tAmount: " +
                    "{1}", tran.When(), tran.Amount());
		}
        Console.WriteLine();
    }
}
\end{lstlisting}  

Файл BankAccount.cs:
\begin{lstlisting}[style=CSharpStyle]
using System.Collections;

class BankAccount : IDisposable
{
	private long accNo;
    private decimal accBal;
    private AccountType accType;
    private Queue tranQueue = new Queue();
    
    private static long nextNumber = 123;

	// Constructors
    public BankAccount()
    {
        accNo = NextNumber();
        accType = AccountType.Checking;
        accBal = 0;
    }

    public BankAccount(AccountType aType)
	{
        accNo = NextNumber();
        accType = aType;
        accBal = 0;
	}

	public BankAccount(decimal aBal)
	{
        accNo = NextNumber();
        accType = AccountType.Checking;
        accBal = aBal;
	}

	public BankAccount(AccountType aType, decimal aBal)
	{
        accNo = NextNumber();
        accType = aType;
        accBal = aBal;
	}

    public bool Withdraw(decimal amount)
    {
        bool sufficientFunds = accBal >= amount;
        if (sufficientFunds) {
            accBal -= amount;
            BankTransaction tran = new BankTransaction(
                    -amount);
            tranQueue.Enqueue(tran);
        }
        return sufficientFunds;
    }
    
    public decimal Deposit(decimal amount)
    {
        accBal += amount;
        BankTransaction tran = new BankTransaction(
                    amount);
        tranQueue.Enqueue(tran);
        return accBal;
    }

	public Queue Transactions()
	{
		return tranQueue;
	}
    
    public long Number()
    {
        return accNo;
    }
    
    public decimal Balance()
    {
        return accBal;
    }
    
    public string Type()
    {
        return accType.ToString();
    }

    private static long NextNumber()
    {
        return nextNumber++;
    }

    public void Dispose()
    {
        tranQueue.Clear();
    }
}
\end{lstlisting}  

Файл BankTransaction.cs:
\begin{lstlisting}[style=CSharpStyle]
using System.IO;
/// <summary>
///   A BankTransaction is created every time a deposit 
///   or withdrawal occurs on a BankAccount
///   A BankTransaction records the amount of money
///   involved, together with the current date and time.
/// </summary>
public class BankTransaction
{
    private readonly decimal amount;
    private readonly DateTime when;

    public BankTransaction(decimal tranAmount)
    {
        Console.WriteLine("Tran created");
        amount = tranAmount;
        when = DateTime.Now;
    }

    public decimal Amount()
    {
        return amount;
    }

    public DateTime When()
    {
        return when;
    }			 

    ~BankTransaction()
    {
        StreamWriter swFile = File.AppendText("Transactions.Dat");
        swFile.WriteLine("Date/Time: {0}\tAmount: {1}", 
                        when, amount);
        swFile.Close();
        GC.SuppressFinalize(this);
    }
}

\end{lstlisting}  

Файл AccountType.cs:
\begin{lstlisting}[style=CSharpStyle]
enum AccountType 
{ 
    Checking, 
    Deposit 
}
\end{lstlisting}  
\endinput
