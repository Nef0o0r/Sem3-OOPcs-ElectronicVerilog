% Библиография оформлена в ручную в соответствии с ГОСТ

\chapter*{
	СПИСОК ИСПОЛЬЗОВАННЫХ ИСТОЧНИКОВ
}
\addcontentsline{toc}{chapter}{СПИСОК ИСПОЛЬЗОВАННЫХ ИСТОЧНИКОВ}
\label{ch:bibla}
% \vspace{-20pt}
{
	\setlist[1]{labelindent=0pt}
	
	\begin{enumerate} [label=\arabic*]
		\item \label{s:1} Халиуллин Р.А. Задание для курсовой работы. — 03.2024. — URL: \url{https://vec.etu.ru/moodle/mod/forum/discuss.php?d=5329#p8197} (дата обращения: 01.06.2024).
		\item \label{s:2} Жадаев А.Г. Антивирусная защита ПК: от «чайника» к пользователю : cамоучитель. — СПб: БХВ-Петербург, 2010. — 224 с.
		\item \label{s:3} Ritchie D.M., Johnson S.C., Lesk M., Kernighan B., [et al.]. The C programming language // Bell Sys. Tech. J. — 1978. — Vol. 57, No. 6. — P. 1991–2019.
		\item \label{s:4} Халиуллин Р.А. Информация об основных стандартных целочисленных типах данных языков программирования C и C++. — 09.2023. — URL: \url{https://vec.etu.ru/moodle/mod/forum/discuss.php?d=4688#p7201} (дата обращения: 01.06.2024).    
	\end{enumerate}
}

\endinput
