 \thispagestyle{empty}

\begin{center}
	Федеральное государственное автономное образовательное учреждение \\
	высшего образования <<Санкт-Петербургский государственный \\ 
	электротехнический университет <<ЛЭТИ>> \\
	им. В.И. Ульянова (Ленина)>>\\
	Кафедра Информационной безопасности\\
\end{center}

\vfill

\begin{center}
	
	{\bfseries ОТЧЁТ \\
	по лабораторной работе №4\\
	по дисциплине <<Электроника и схемотехника>>\\
	Тема: <<Использование семисегментного индикатора>>}
\end{center}

\vfill

\begin{table}[H]
	\tabcolsep = 0pt
	\extrarowheight = 18pt
	% таблица сделана в соответствии с шаблоном - если не влезает фамилия, то рекомендуется уменьшить ширину первого столбца
	\begin{tabularx}{\textwidth}{>{\raggedright\arraybackslash}b{6.3cm}>{\raggedright\arraybackslash}b{4.6cm}>{\centering\arraybackslash}X}
		Студент гр. 3363&&Минко Д.А.\\ \cline{2-2}
		Студент гр. 3363&&Гончаренко О.Д.\\ \cline{2-2}
		Студент гр. 3363&&Овсейчик Н.И.\\ \cline{2-2}
		Преподаватель&&Рыбин В.Г.\\ \cline{2-2}
	\end{tabularx}
\end{table}

\begin{center}
	Санкт-Петербург \\ 2024
\end{center}
